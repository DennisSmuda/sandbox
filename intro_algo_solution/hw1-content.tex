
\problem{2.3-3}

Use mathematical induction to show that when $n$ is an exact power of 2, the solution of the recurrence

\begin{equation*}
T(n) = \left\{
  \begin{array}{ll}
    2         & \text{if $n=2$,}\\
    2T(n/2)+n & \text{if $n=2^k$, for $k>1$}
  \end{array}
\right.
\end{equation*}

is $T(n)=n\lg n$.

\proof

Assume $n = 2^k$, we perform mathematical induction on $k$. When $k =1$, we have $T(2^k) = T(2) = 2 = 2\times \lg 2$, the recurrence holds true.
If the recurrence is true for $ k \leq k_0$, then for $k = k_0 + 1$, we have:

\begin{eqnarray*}
T(2^{k_0 + 1}) &=& 2T\left(\frac{2^{k_0 + 1}}{2}\right) + 2^{k_0 + 1}\\
&=& 2T(2^{k_0}) + 2^{k_0 + 1}\\
&=& 2\times 2^{k_0} \times k_0 + 2^{k_0 + 1}\\
&=& 2^{k_0 + 1} \times (k_0 + 1),
\end{eqnarray*}

the solution is still true. So by mathematical induction, we have proved the recurrence.
\qed

\problem{2.3-6}

Observe that the \textbf{while} loop of lines 5-7 of the \textsc{Insertion-Sort} procedure below uses a linear search to scan (backward)
through the sorted subarray $A[i..j-1]$. Can we use a binary search instead to improve the overall worst-case running
time of insertion sort to $\Theta(n\lg n)$?

\begin{algorithm}[H]
\caption{\textsc{Insertion-Sort}}
\For{$j \leftarrow 2$ \KwTo $length[A]$}{
  $key \leftarrow A[j]$\\
  $\rhd$ Insert $A[j]$ into the sorted sequence $A[1..j-1]$.\\
  $j \leftarrow j - 1$\\
  \While {$i > 0$ and $A[i] > key$} {
    $A[i + 1] \leftarrow A[i]$ \\
    $i \leftarrow i -1 $
  }
  $A[i + 1] \leftarrow key$
}
\end{algorithm}


\answer

It is not possible. In each round of the \textsc{Insertion-Sort}, the algorithm first \textbf{finds the point to insert} and then \textbf{insert data},
each has cost of $\Theta(n)$. By using binary search, we could reduce the time for \textbf{finding the point to insert} to $\Theta(\lg n)$,
but it still has cost of $\Theta(n)$  to \textbf{insert data} in worst case. So the overall cost is still $n\times \Theta(n) = \Theta(n^2)$.
\qed

\problem{2-4 Inversions}

Let $A[1..n]$ be an array of $n$ distinct numbers. If $i<j$ and $A[i]>A[j]$, then the pair $(i,j)$ is called an \textbf{inversion} of $A$.

\begin{description}
\item[a. \hspace{9pt}] List the five inversions of the array $\langle2, 3, 8, 6, 1\rangle$.

\item[b. \hspace{9pt}] What array with elements from the set ${1, 2, ..., n}$ has the most inversions? How many does it have?

\item[c. \hspace{9pt}] What is the relationship between the running time of insertion sort and the number of inversions in the input array? Justify your answer.

\item[d. \hspace{9pt}] Give an algorithm that determines the number of inversions in an permutation on $n$ elements in $\Theta(n\lg n)$ worst-case time. (Hint: Modify merge sort.)
\end{description}

\answer

\begin{description}
\item[a. \hspace{9pt}] $(1, 5), (2, 5), (3, 5), (4, 5), (3, 4)$.

\item[b. \hspace{9pt}] The array $[n, n -1, n - 2, ..., 2, 1]$ has most inversions. It has ${{n}\choose{2}} = \frac{n(n - 1)}{2}$ inversions.

\item[c. \hspace{9pt}] The running time of insertion sort is \textbf{linear corelational} to the number of inversions in the input array.
This is true, because each time an element is moved, the number of inversion pairs is reduced by one. So the number of move action is equal to
the number of inversion pairs, and since move action requires constant time, we know the conlusion is true.

\item[d. \hspace{9pt}] Sorted arrays have 0 inversion pairs. Given 2 sorted array $A=(a_0, a_1, ..., a_m)$ and $B = (b_0, b_1, ..., b_n)$,
and let $\overline{AB} = (a_0, a_1, ..., a_m, b_0, b_1, ..., b_n)$. The number of inversions in $\overline{AB}$ could be caculated by the following method:

For any $a_i$ in $\overline{AB}$, suppose we could find such an $j$ that $b_j < a_i < b_{j + 1}$, then we know that there are $j$ numbers in $\overline{AB}$
which are smaller than $a_i$, introducing $j$ inversion pairs. Otherwise we cannot find a proper $j$, we know that there is no inversion pair for $a_i$.
For any $b_j$ in $\overline{AB}$ we have similar results.

So we could write the following algorithm for counting inversion pairs, similar to \textsc{Merge-Sort}, it has a complexity of $\Theta(n\lg n)$:

\begin{algorithm}[H]
\caption{\textsc{Inversion-Pair}}
\SetKwInOut{Input}{input}
\SetKwInOut{Output}{output}
\Input{$A$: an array, $p$ \& $q$: index for array $A$}
\Output{The number of inversion pairs in $A[p...q]$, inclusive}
\If{$ p\geq q$}{
  \KwRet 0\tcp*[r]{at most 1 item, there is no inversion}
}
\Else{
  \tcp{$S$: number of inversion pairs}
  $S \leftarrow  $ \textsc{Inversion-Pair}$(A, p, k)$ + \textsc{Inversion-Pair}$(A, k + 1, q)$\\
  $i \leftarrow 1$, $j \leftarrow 1, k \leftarrow \lfloor\frac{p + q}{2}\rfloor$\\
  $L \leftarrow A[p...k], R \leftarrow A[k + 1...q]$\\
  clear array $A$\\
  \While{$i < length(L)$ and $j < length(R)$}{
    \If{$L_i < R_j$}{
      append $L_i$ to $A$\\
      $i \leftarrow i + 1$\\
    }
    \ElseIf {$L_i > R_j$}{
      append $R_j$ to $A$\\
      $S \leftarrow S + (length(L) - i + 1)$\\
      $j \leftarrow j + 1$\\
    }
    \Else {
      \tcp{$L_i = R_j$}
      append $L_i$ and $R_j$ to $A$\\
      $i \leftarrow i + 1$\\
      $j \leftarrow j + 1$\\
      }
  }
  $S \leftarrow S +  length(R) \times (length(L) - i + 1)$\\
  append the rest of $L$ to $A$\\
  append the rest of $R$ to $A$\\
  \KwRet $S$
}
\end{algorithm}

\end{description}
\qed


\problem{3.2-3} Prove equation (3.18). Also prove that $n! = \omega(2^n)$ and $n! = o(n^n)$.

\begin{equation}
\lg(n!) = \Theta(n\lg n)
\tag{3.18}
\end{equation}

\proof

With Stirling's equation:

\begin{eqnarray*}
\lg (n!) &=& \lg\left(\sqrt{2\pi n}\left(\frac{n}{e}\right)^n\left(1 + \Theta\left(\frac{1}{n}\right)\right)\right)\\
&=& \frac{1}{2}\lg n + n(\lg n - \lg e) + \Theta(1)\\
&=& n \lg n + \Theta(n)\\
&=& \Theta(n \lg n).
\end{eqnarray*}

Since $\lg 2^n = n = o(n\lg n)$, we have $n ! = \omega(2^n)$.

And by Stirling's equation:

\begin{eqnarray*}
n! &=& \sqrt{2\pi n}\left(\frac{n}{e}\right)^n\left(1 + \Theta\left(\frac{1}{n}\right)\right)\\
&<& n^n \times \frac{\sqrt{2\pi n}}{e^n} \times 2\\
&=& n^n \times \omega(1),
\end{eqnarray*}

So we have $n! = o(n^n)$.
\qed

\problem{3-3 Ordering by asymptotic growth rates}

\begin{description}
\item[a. \hspace{9pt}] Rank the following functions by order of growth; that is, find an arrangement $g_1, g_2, ..., g_{30}$ of the functions satisfying
$g_1 = \Omega(g_2), g_2 = \Omega(g_3), ..., g_{29} = \Omega(g_{30})$. Partition your list info equivalence classes such that $f(n)$ and $g(n)$
are in the same class if and only if $f(n) = \Theta(g(n))$.

\begin{equation*}
\begin{array}{cccccc}
\lg(\lg^*n)                 & 2^{\lg^*n}          & \left(\sqrt 2\right)^{\lg n}  & n^2           & n!        & (\lg n)!\\
\left(\frac{3}{2}\right)^n  & n^3                 & \lg^2n                        & \lg(n!)       & 2^{2^n}   & n^{\frac{1}{\lg n}}\\
\ln\ln n                    & lg^*n               & n \cdot 2^n                   & n^{\lg\lg n}  & \ln n     & 1\\
2^{\lg n}                   & (\lg n)^{\lg n}     & e^n                           & 4^{\lg n}     & (n+1)!    &  \sqrt{\lg n}\\
\lg^*(\lg n)                & 2^{\sqrt{2 \lg n}}  & n                             & 2^n           & n\lg n    & 2^{2^{n+1}}
\end{array}
\end{equation*}

\item[b. \hspace{9pt}] Give an example of a single nonnegative function $f(n)$ such that for all functions $g_i(n)$ in part (a), $f(n)$ is neither $O(g_i(n))$
nor $\Omega(g_i(n))$.
\end{description}

\answer

\begin{description}

\item[a. \hspace{9pt}] First of all, we have 
$$n^{\frac{1}{\lg n}} = n^{\log_n{2}} = 2,$$

which means $n^{\frac{1}{\lg n}} = \Theta(1).$ Constant numbers has zero growth, so they are in the lowest class.

Let $f(x) = 2^x$, then we have $\log^*{\left(f^{(t)}(1)\right)} = t$ and $\lg\left(f^{(t)}(1)\right) = \lg\left(2^{f^{(t-1)}(1)}\right) = f^{(t-1)}(1)$. Now assume $n = f^{(t)}(1)$, we have:

\begin{eqnarray*}
\lg\left(\lg^*n\right) &=& \lg\left(\lg^*\left(f^{(t)}(1)\right)\right) = \lg t\\
\lg^*\left(\lg n\right) &=& \lg^*\left(\lg \left(f^{(t)}(1)\right)\right) = \lg^*\left(f^{(t-1)}(1)\right) = t-1\\
\lg^*(n) &=& \lg^*\left(f^{(t)}(1)\right) = t\\
2^{\lg^*\left(\lg n\right)} &=& 2^t\\
\ln\ln n &=& \ln \left(\ln \left(f^{(t)}(1)\right)\right) = \Theta\left(f^{(t - 2)}(1)\right).
\end{eqnarray*}

Since $\lg t < t - 1 \sim t < 2^t <\Theta\left(f^{(t - 2)}(1)\right)$ when $t$ is big enough, we know that:

\begin{equation}
\lg\left(\lg^*n\right) < \lg^*\left(\lg n\right) \sim \lg^*n < 2^{\lg^*n} < \ln \ln n.
\end{equation}

Now assume $n = 2^t$, we have:

\begin{eqnarray*}
\ln\ln n &=& \ln \ln \left(2^t\right) = \Theta\left(\lg t\right)\\
\sqrt{\lg n} &=& \sqrt{t}\\
\ln n &=& \Theta(t)\\
\lg^2{n} &=& \left(\lg2^t\right)^2 =  t^2 = 2^{2\lg t}\\
2^{\sqrt{2\lg n}} &=& 2^{\sqrt{2t}}\\
\left(\sqrt{2}\right)^{\lg n} &=& 2^{\frac{t}{2}}\\
n &=& 2^t.
\end{eqnarray*}

From those equations we have:

\begin{equation}
\ln \ln n  < \sqrt{\lg n} < \ln n < \lg^2 n < 2^{\sqrt{2\lg n}} < \left(\sqrt{2}\right)^{\lg n}  < n.
\end{equation}

And it is easy to sort those functions:

\begin{equation}
n = 2^{\lg n} < n \lg n \sim \lg\left(n !\right) < n^2 = 4^{\lg n} < n^3,
\end{equation}

with the fact that:

\begin{eqnarray*}
\lg (n!) &=& \lg \left(\sqrt{2\pi n} \left(\frac{n}{e}\right)^n\left(1 + \Theta\left(\frac{1}{n}\right)\right)\right)\\
&=& \frac{1}{2}\ln n + n(\ln n - 1) + \Theta(1)\\
&=& \Theta(n \ln n).
\end{eqnarray*}

It is trivial to prove that $8^t < t!$. Replace $t$ with $\lg n$, we have

$$8^{\lg n} = n^3 < (\lg n)!.$$

To compare $(\lg n)!$ and $n^{\lg \lg n}$,  we have:

\begin{eqnarray*}
\lg \left((\lg n)! \right) &=& \frac{1}{2} \ln \lg n + \lg n \times (\ln \lg n - 1) + \Theta(1)\\
&=& \frac{1}{\lg e}\times \lg n \times \lg \lg n + \Theta(\lg n)\\
\lg \left(n^{\lg \lg n}\right) &=& \lg n \times \lg \lg n,
\end{eqnarray*}

which means:

$$(\lg n)! < n^{\lg \lg n}.$$

To compare $ n^{\lg \lg n}$ and $(\lg n)^{\lg n}$, we have:

\begin{eqnarray*}
\lg \left(n ^ {\lg \lg n}\right) &=& \lg n \times \lg \lg n\\
\lg \left(\left(\lg n\right)^{\lg n}\right) &=& \lg n \times \lg \lg n,
\end{eqnarray*}

which means:

$$ n^{\lg \lg n}  = (\lg n)^{\lg n}.$$

To compare $(\lg n)^{\lg n}$ and $\left(\frac{3}{2}\right)^n$, we have:

\begin{eqnarray*}
\lg \left( \frac{3}{2}\right)^n &=& n\times \lg \left(\frac{3}{2}\right) > \lg^2 n > \lg n\times \lg \lg n = \lg \left(\left(\lg n\right)^{\lg n}\right),
\end{eqnarray*}

which means:

$$ (\lg n)^{\lg n} < \left(\frac{3}{2}\right) ^ n.$$

And it is easy to sort these functions:

$$ \left(\frac{3}{2}\right)^n < 2^n < n \cdot 2^n < e^n < ( n^n  <) n!  < (n + 1)!.$$

To compare $(n + 1)!$ and $2^{2^n}$, we have:

\begin{eqnarray*}
\lg (n + 1)! &=& \Theta(n \lg n)\\
\lg \left(2^{2^n}\right) &=& 2^n,
\end{eqnarray*}

which means $(n + 1)! < 2^{2^n}$.

And finally we have $2^{2^n} < 2^{2^{n + 1}}$.

Now we could conclude the problem:

\begin{eqnarray*}
1 \sim n^{\frac{1}{\lg n}} < \lg \left(\lg^*n\right) < \lg^*\left(\lg n\right) \sim \lg^* n < 2^{\lg^* n} < \ln\ln n < \sqrt{\lg n} < \ln n < \lg^2 n\\
< 2^{\sqrt{2\lg n}} < \left(\sqrt 2\right) ^ {\lg n} < n = 2^{\lg n}  < n\lg n \sim \lg \left(n!\right) < n^2 = 4^{\lg n}< n^3  < (\lg n)!\\
< n^{\lg \lg n} = \left(\lg n \right)^{\lg n} < \left(\frac{3}{2}\right)^n < 2^n < n\cdot 2^n < e^n < n! < (n + 1)! < 2^{2^n} < 2^{2^{n + 1}}
\end{eqnarray*}

\item[b. \hspace{9pt}] This function is a solution:

$$(1 + \sin(n)) \times 3^{3^n}.$$

\end{description}
\qed


