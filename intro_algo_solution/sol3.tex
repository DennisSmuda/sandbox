\documentclass[a4paper,10pt]{article}

%A Few Useful Packages
\usepackage{marvosym}
\usepackage{fontspec} 					%for loading fonts
\usepackage{xunicode,xltxtra,url,parskip} 	%other packages for formatting
\RequirePackage{color,graphicx}
\usepackage[usenames,dvipsnames]{xcolor}
\usepackage[big]{layaureo} 				%better formatting of the A4 page
% an alternative to Layaureo can be ** \usepackage{fullpage} **
\usepackage{longtable} 				%for experience
\usepackage{titlesec}					%custom \section
\usepackage{amsmath}
\usepackage{amsfonts}
\usepackage[vlined,linesnumbered,ruled]{algorithm2e}

%Setup hyperref package, and colours for links
\usepackage{hyperref}
\definecolor{linkcolour}{rgb}{0,0.2,0.6}
\hypersetup{colorlinks,breaklinks,urlcolor=linkcolour, linkcolor=linkcolour}

%FONTS
\defaultfontfeatures{Mapping=tex-text}
\setmainfont[SmallCapsFont = Fontin SmallCaps]{Fontin}

%CV Sections inspired by: 
%http://stefano.italians.nl/archives/26
\titleformat{\section}{\Large\scshape\raggedright}{}{0em}{}[\titlerule]
\titlespacing{\section}{0pt}{3pt}{3pt}

%tweak page height
\addtolength{\topmargin}{-.125in}
\addtolength{\textheight}{0.25in}

%-------------WATERMARK TEST [**not part of a CV**]---------------
\usepackage[absolute]{textpos}

\setlength{\TPHorizModule}{30mm}
\setlength{\TPVertModule}{\TPHorizModule}
\textblockorigin{2mm}{0.65\paperheight}
\setlength{\parindent}{0pt}

\newcommand{\problem}[1]{\section*{Problem #1}}
\newcommand{\answer}{\paragraph{Answer:}}
\newcommand{\proof}{\paragraph{Proof:}}
\newcommand{\qed}{\hfill \ensuremath{\Box}}
\newcommand{\todo}{\textcolor{red}{TODO}{} }

% TODO
\newcommand{\nless}{\textcolor{red}{TODO!!!}{} }

%--------------------BEGIN DOCUMENT----------------------
\begin{document}

%WATERMARK TEST [**not part of a CV**]---------------
\font\wm=''Baskerville:color=787878'' at 8pt
\font\wmtoday=''Baskerville:color=FF1493'' at 8pt
{\wm 
	\begin{textblock}{1}(0,0)
		\rotatebox{-90}{\parbox{500mm}{
			Typeset by Yang Zhang with \XeTeX\ on {\wmtoday \today}
		}
	}
	\end{textblock}
}

\pagestyle{empty} % non-numbered pages

\font\fb=''[cmr10]'' %for use with \LaTeX command

%--------------------TITLE-------------
\par{\centering
  {\Large Algorithms and Complexity Analysis}
  \\
  { Instructed by Yin ZHAO, Spring 2011}
  \\
		{\Huge Solution for homework 3
	}\bigskip\par}

%--------------------SECTIONS-----------------------------------

\problem{6.3-2}
Why do we want the loop index $i$ in line 2 of \textsc{Build-Max-Heap} to decrease from 
$\left\lfloor \frac{length[A]}{2}\right\rfloor$ to 1 rather than increase from 1
to $\left\lfloor \frac{length[A]}{2}\right\rfloor$?

\begin{algorithm}[H]
\caption{\textsc{Build-Max-Heap}$(A)$}
$heap$-$size[A]\leftarrow length[A]$\\
\For{$i \leftarrow \left\lfloor\frac{length[A]}{2}\right\rfloor$ \textnormal{\textbf{downto}} $1$} {
  \textsc{Max-Heapify}$(A, i)$
}
\end{algorithm}

\answer

This is because we must make sure that the smaller subtrees are changed to heaps first, so that the
algorithm could build bigger subtrees correctly. If we increase the index $i$ from 1 to
$\left\lfloor\frac{length[A]}{2}\right\rfloor$, the heap property cannot be correctly preserved.
\qed

\problem{7-1 Hoare partition correctness}
The version of \textsc{Partition} given in this chapter is not the original partitioning algorithm. 
Here is the original parition algorithm, which is due to T.Hoare:

\begin{algorithm}[H]
\caption{\textsc{Hoare-Partition}(A, p, r)}
$x \leftarrow A[p]$\\
$i \leftarrow p - 1$\\
$j \leftarrow r + 1$\\
\While{$true$}{
  \Repeat{$A[j] \leq x$}{
    $j \leftarrow j - 1$\\
  }
  \Repeat{$A[i] \geq x$}{
    $i \leftarrow i + 1$\\
  }
  \If{$i < j$}{
    exchange $A[i] \leftrightarrow A[j]$
  }\Else{
    \Return $j$
  }
}
\end{algorithm}

\begin{description}
\item[a. \hspace{9pt}] Demonstrate the operation of \textsc{Hoare-Partition} on the array 
$A=\langle13, 19, 9, 5, 12, 8, 7, 4, 11,\\ 2, 6, 21\rangle$, showing the values of
the array and auxiliary values after each iteration of the \textbf{while} loop in lines 4\textasciitilde 11.\\

\end{description}
The next three questions ask you to give a careful argument that the procedure \textsc{Hoare-Partition} is correct.
Prove the following:

\begin{description}
\item[b. \hspace{9pt}] The indices $i$ and $j$ are such that we never access an element of $A$ outside
the subarray $A[p\ldots r]$.

\item[c. \hspace{9pt}] When \textsc{Hoare-Partition} terminates, it returns a value $j$ such that $p\leq j < r$.

\item[d. \hspace{9pt}] Every element of $A[p\ldots j]$ is less than or equal to every element of
$A[j + 1\ldots r]$ when \textsc{Hoare-Partition} terminates.
\end{description}

The \textsc{Partition} procedure shown below separates the pivot value (originally in $A[r]$) from the
two partitions it forms. The \textsc{Hoare-Partition} procedure,
on the other hand, always places the pivot value (originally in $A[p]$) into one of the two paritions
$A[p\ldots j]$ and $A[j + 1\ldots r]$. Since $p\leq j < r$, this
split is always nontrivial.

\begin{algorithm}[H]
\caption{\textsc{Partition}(A, p, r)}
$x\leftarrow A[r]$\\
$i \leftarrow p - 1$\\
\For{$j \leftarrow p$ \KwTo $r-1$}{
  \If{$A[j] \leq x$} {
    $i \leftarrow i + 1$\\
    exchange $A[i] \leftrightarrow A[j]$\\
  }
}
exchange $A[i + 1]\leftrightarrow A[r]$\\
\Return {$i + 1$}
\end{algorithm}

\begin{description}
\item[e. \hspace{9pt}] Rewrite the \textsc{Quicksort} procedure to use \textsc{Hoare-Partition}.
\end{description}

\answer

\begin{description}
\item[a. \hspace{9pt}] The iteration is shown in the following table:

\begin{center}
\begin{tabular}{|c|c|c|c|}
\hline
Loop No. & $A$ & $i$ & $j$\\\hline
1 & $\langle 6, 19, 9, 5, 12, 8, 7, 4, 11, 2, 13, 21\rangle$ & 1 & 11\\\hline
2 & $\langle 6, 2, 9, 5, 12, 8, 7, 4, 11, 19, 13, 21\rangle$ & 2 & 10\\\hline
3 & $\langle 6, 2, 9, 5, 12, 8, 7, 4, 11, 19, 13, 21\rangle$ & 10 & 9\\\hline
\end{tabular}
\end{center}

\item[b. \hspace{9pt}] First of all, $j$ was given initial value $r + 1$, and it is decreased before
the access of $A[j]$, so $j\leq r$ when $j$ is used as an index to $A$.
Similarly we have $i \geq p$.

Suppose that when the procedure \textbf{return}s, there is a $k$ such that $j < k < i$.
Since the \textbf{repeat} statement of lines $5 \sim 7$ halts only when $A[j] \leq x$, 
we know that $A[k] > x$. Similarly, the \textbf{repeat} statement of lines 8\textasciitilde 10
halts only when $A[j] \geq x$, which requires $A[k] < x$. From this contradiction we knows
that the $k$ does not exist, which means $j \geq i - 1$.

On each round of the \textbf{while} loop, line 6 and line 9 are both executed at least once.
If the \textbf{while} loop is executed more than once, we have $j \leq r - 1$ and
$i \geq p + 1$. With the fact that $j \geq i -1$, it is trivial to prove that 
$p \leq i \leq r$ and $p \leq j \leq r$.


Now we consider the case when the \textbf{while} loop is executed only once. The
\textbf{repeat} statement of lines 5\textasciitilde 7 halts when $A[j] \leq x$. With the fact that $A[p] = x$,
we have $j \geq p$. And the \textbf{repeat} statement of lines 8\textasciitilde 10 will halt with $i = p$.
Now we have proved that under all cases, $p \leq i \leq r$ and $p \leq j \leq r$, 
so we never access an element of $A$ outside $A[p\ldots r]$.

\item[c. \hspace{9pt}] The original question should add another requirement that $p < r$. We have
already proved that $p \leq j \leq r$. Suppose the procedure returned $j = r$, then
we know that the \textbf{while} loop is executed only once (in each \textbf{while} loop, $j$ is
decreased at least by 1). And we know that on the first round of the \textbf{while}
loop, the \textbf{repeat} statement of lines $8 \sim 10$ will halt with $i = p$. The \textbf{return}
statement will be reached only when $i \geq j$, that is, $p \geq r $. Now we have
a contradiction here, which means $j \neq r$, so the procedure returns $p \leq j < r$.

\item[d. \hspace{9pt}] We prove the invariant that ``after line 10, every element in $A[p \ldots i - 1]$
is less than or equal to $x$, and every element in $A[j + 1 \ldots r]$ is greater than
or equal to $x$''.

Before the first round of the \textbf{while} loop, this invariant is true, since there is no element in the
subarrays. Suppose the invariant is true before the $k$th round, then in the $k$th
round, the \textbf{repeat} statement in lines $5 \sim 7$ decrease $j$ until $A[j] \leq x$, so it is still
true that any element in $A[j + 1 \ldots r]$ is greater than or equal to $x$.
Similarly we know that any element in $A[p \ldots i - 1]$ is less than or equal to $x$. Lines 11\textasciitilde 14
 does not change the value of $i$ and $j$, so $i$ and $j$ remains unchanged until
the $(k + 1)$th loop. By mathematical induction we know that the invariant is true.

The procedure returns only when $i \geq j$, and we have proved $j \geq i - 1$ in question (c), so we
have either $i = j$ or $i = j + 1$. If $i = j$, since $A[i] \geq x$ and $A[j] \leq x$, 
we have $A[i] = A[j] = x$, so every element in $A[p \ldots j]$ is less than or equal to $x$,
and every element in $A[j + 1 \ldots r]$ is greater than or equal to $x$. If $i = j + 1$,
we have the same conclusion. Thus the statement of question (d) is proved.

\item[e. \hspace{9pt}] The algoritm is given below.

\end{description}
\begin{algorithm}[H]
\caption{\textsc{Hoare-Quicksort}(A, p, r)}
\If{$p \leq r$}{
  \Return
}\Else{
  $j \leftarrow$\textsc{Hoare-Partition}$(A, p, r)$\\
  \textsc{Hoare-Quicksort}$(A, p, j)$\\
  \textsc{Hoare-Quicksort}$(A, j + 1, r)$
}
\end{algorithm}
\qed


\problem{7-4}
\todo

\problem{8.2-4}
Describe an algorithm that, given $n$ intergers in the range 0 to $k$, preprocesses its input and then answers
any query about how many of the $n$ integers fall into a range
$[a\ldots b]$ in $O(1)$ time. Your algorithm should use $\Theta(n + k)$ preprocessing time.
\answer

The algorithm is given below. The \textsc{Preprocess} returns an array, which will be used by \textsc{Count-Interval}.

\begin{algorithm}[H]
\caption{\textsc{Preprocess}$(A, n, k)$}
Create array $C$ with size $k + 1$, index starts at $0$\\
\For{$i = 0$ to $k$}{
  $C[i] \leftarrow 0$
}
\For{$i = 1$ to $n$}{
  $C[A[i]] \leftarrow C[A[i]] + 1$
}
\tcp{Now $C[j]$ contains number of elements with value $j$}
\For{$i = 1$ to $k$}{
  $C[A[i]] \leftarrow C[A[i]] + C[A[i - 1]]$
}
\tcp{Now $C[j]$ contains number of elements in the range $[0, j]$}
\Return $C$
\end{algorithm}

\begin{algorithm}[H]
\caption{\textsc{Count-Interval}$(C, a, b)$}
\tcp{Array $C$ is returned by \textsc{Preprocess}$(A, n)$}
\If{$a \neq 0$}{
  \Return $C[b] - C[a - 1]$
}\Else{
  \Return $C[b]$
}
\end{algorithm}
\qed



\problem{8-6}
\todo



\problem{9.3-1}
In the algorithm \textsc{Select}, the input elements are divided into groups of 5.
Will the algorithm work in linear time if they are divided into groups of 7? Argue that
\textsc{Select} does not run in linear time if groups of 3 are used.
\answer

By similar analyze, we know that the smaller part has at least

$$ 4\left(\left\lceil\frac{1}{2}\left\lceil\frac{n}{7}\right\rceil\right\rceil-2\right)\geq \frac{2n}{7} - 8.$$

So, in worst case, \textsc{Select} is recursively running on $\frac{5n}{7} + 8$ elements. Now we obtain the recurrence:

\begin{equation*}
T(n) \leq \left\{
  \begin{array}{ll}
    \Theta(1)     & \text{if $n \leq 140$,}\\
    T\left(\left\lceil\frac{n}{7}\right\rceil\right) + T\left(\frac{5n}{7} + 8\right) + O(n) & \text{if $n > 140$.}
  \end{array}
\right.
\end{equation*}

By substitution we have:

\begin{eqnarray*}
T(n)  &\leq& c\left(\left\lceil\frac{n}{7}\right\rceil\right) + c\left(\frac{5n}{7} + 8\right) + bn\\
&\leq& \frac{cn}{7} + \frac{5cn}{7} + 8c + bn\\
&=&cn + \left(8c + bn - \frac{cn}{7}\right)
\end{eqnarray*}

Now we only need $8c + \left(b - \frac{c}{7}\right)n \leq 0$, which could be satisfied by choosing $c > 21b$.

If groups of 3 is chosen, then we would smaller part with at least

$$2\left(\left\lceil\frac{1}{2}\left\lceil\frac{n}{3}\right\rceil\right\rceil -2\right) \geq \frac{n}{3} - 4.$$

So in worst case, the algorithm will run on $\frac{2n}{3} + 4$ elements recursively. And we will have the recurrence:

\begin{eqnarray*}
T(n) &=& T\left(\left\lceil\frac{n}{3}\right\rceil\right) + T\left(\frac{2n}{3} + 4\right) + O(n)\\
&\geq& T\left(\frac{n}{3}\right) + T\left(\frac{2n}{3} + 4\right) + O(n)\\
\end{eqnarray*}

Since $\frac{1}{3} + \frac{2}{3} = 1 \nless 1$,  we cannot prove that the algorithm runs in linear time.
\qed
\end{document}

\todo the nless
