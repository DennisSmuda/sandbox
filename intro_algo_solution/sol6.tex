\documentclass[a4paper,10pt]{article}

%A Few Useful Packages
\usepackage{marvosym}
\usepackage{fontspec} 					%for loading fonts
\usepackage{xunicode,xltxtra,url,parskip} 	%other packages for formatting
\RequirePackage{color,graphicx}
\usepackage[usenames,dvipsnames]{xcolor}
\usepackage[big]{layaureo} 				%better formatting of the A4 page
% an alternative to Layaureo can be ** \usepackage{fullpage} **
\usepackage{longtable} 				%for experience
\usepackage{titlesec}					%custom \section
\usepackage{amsmath}
\usepackage{amsfonts}
\usepackage[vlined,linesnumbered,ruled]{algorithm2e}

%Setup hyperref package, and colours for links
\usepackage{hyperref}
\definecolor{linkcolour}{rgb}{0,0.2,0.6}
\hypersetup{colorlinks,breaklinks,urlcolor=linkcolour, linkcolor=linkcolour}

%FONTS
\defaultfontfeatures{Mapping=tex-text}
\setmainfont[SmallCapsFont = Fontin SmallCaps]{Fontin}

%CV Sections inspired by: 
%http://stefano.italians.nl/archives/26
\titleformat{\section}{\Large\scshape\raggedright}{}{0em}{}[\titlerule]
\titlespacing{\section}{0pt}{3pt}{3pt}

%tweak page height
%\addtolength{\topmargin}{-.125in}
%\addtolength{\textheight}{0.25in}

%-------------WATERMARK TEST [**not part of a CV**]---------------
\usepackage[absolute]{textpos}

\setlength{\TPHorizModule}{30mm}
\setlength{\TPVertModule}{\TPHorizModule}
\textblockorigin{2mm}{0.65\paperheight}
\setlength{\parindent}{0pt}

\newcommand{\problem}[1]{\section*{Problem #1}}
\newcommand{\answer}{\paragraph{Answer:}}
\newcommand{\proof}{\paragraph{Proof:}}
\newcommand{\qed}{\hfill \ensuremath{\Box}}
\newcommand{\todo}{\textcolor{red}{TODO}{} }

% TODO
\newcommand{\nless}{\textcolor{red}{TODO!!!}{} }

%--------------------BEGIN DOCUMENT----------------------
\begin{document}

%WATERMARK TEST [**not part of a CV**]---------------
\font\wm=''Baskerville:color=787878'' at 8pt
\font\wmtoday=''Baskerville:color=FF1493'' at 8pt
{\wm 
	\begin{textblock}{1}(0,0)
		\rotatebox{-90}{\parbox{500mm}{
			Typeset by Yang ZHANG with \XeTeX\ on {\wmtoday \today}
		}
	}
	\end{textblock}
}

\pagestyle{empty} % non-numbered pages

\font\fb=''[cmr10]'' %for use with \LaTeX command

%--------------------TITLE-------------
\par{\centering
  {\Large Algorithms and Complexity Analysis}
  \\\vspace{0.5em}
  { Instructed by Yin ZHAO, Spring 2011, Tsinghua University}
  \\\vspace{1.5em}
	{\Huge Solutions for homework 6\vspace{1em}
	}\bigskip\par}

%--------------------SECTIONS-----------------------------------

\problem{17.1-1}

If the set of stack operations included a \texttt{MULTIPUSH} operation, which pushes k items onto the stack,
would the $O(1)$ bound on the amortized cost of stack operations continue to hold?

\answer

\todo

\qed

\problem{17.2-2}

A sequence of $n$ operations is performed on a data structure. The $i$\textsuperscript{th} operation costs $i$ if $i$ is an exact power of 2, and 1 otherwise. 
Use accounting method of analysis to determined the amortized cost per operation.


\answer

\todo

\qed


\problem{17.3-2}

A sequence of $n$ operations is performed on a data structure. The $i$\textsuperscript{th} operation costs $i$ if $i$ is an exact power of 2, and 1 otherwise. 
Use potential method of analysis to determined the amortized cost per operation.


\answer

\todo

\qed

\problem{17.3-6}

Show how to implement a queue with two ordinary stacks so that the amortized cost of each \texttt{ENQUEUE} and each \texttt{DEQUEUE} operation is $O(1)$.

\answer

\todo

\qed

\problem{19.2-2}

Show the binomial heap that results when a node with key 24 is inserted into the binomial heap shown in the following figure.

\answer

\todo figure

\qed


\end{document}


