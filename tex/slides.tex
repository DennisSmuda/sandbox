\documentclass[ignorenonframetext]{beamer}

\mode<article> {
	\usepackage{fullpage}
	\usepackage{pgf}
	\usepackage{hyperref}
}

\mode<presentation> {
	\usepackage{graphicx}
	\usetheme{Goettingen}
		%AnnArbor Antibes Bergen Berkeley Berlin Boadilla CambridgeUS Copenhagen Darmstadt (*)Dresden Frankfurt (*)Goettingen (*)Hannover
		%Ilmenau JuanLesPins Luebeck Madrid Malmoe Marburg (*)Montpellier PaloAlto (*)Pittsburgh Rochester (*)Singapore Szeged Warsaw
	\setbeamercovered{transparent}
}

\usepackage[latin1]{inputenc}
\usepackage[english]{babel}

%\newcommand{\red}{0.00,0.00,0.00}{}
\newcommand{\red}[1]{\textcolor[rgb]{1, 0, 0}{#1}}
\newcommand{\green}[1]{\textcolor[rgb]{0, 1, 0}{#1}}
\newcommand{\blue}[1]{\textcolor[rgb]{0, 0, 1}{#1}}


\title{A Shallow Understanding of SOM-SD}
\author{Zhang Yang}
\subject{Presentation Programs}%
\institute[Tsinghua University]{Tsinghua University, CST 53}

\begin{document}

\frame{
	\maketitle
}

\section{The Need of SOM-SD}

\frame{
	\frametitle{The Need of SOM-SD}

	\begin{itemize}
		\item SOM-SD: \blue{Self-Organized Map of Structured Data}
		\pause
	\item Traditional SOM cannot handle Structured Data, such as Trees and Graphs, which is extensively useful in Computer Science
		\pause
	\item M.Hagenbuchner, A.Sperduti, and A.C Tsoi introduced SOM-SD
		\pause
	\item The SOM-SD, which is an enhanced version of SOM, is capable of handling directed acyclic graphs(DAGs)
	\end{itemize}
}

\section{Brief Revision of SOM}
\frame {
	\frametitle{Brief Revision of SOM}

	SOM is based upon competitive learning, and adopts a winner-take all policy. It organizes a region $\mathcal{R}$ in $\mathbb{R}^n$, where
	$n$ is the dimension of the SOM net, and often chosen to be 2. Each point in $\mathcal{R}$ represents a vector in $\mathbb{R}^m$, where
	$m$ is the dimension of the input vector. When input $x$ was given to the SOM, a winner $i(x) \in D$ that most simmilar to $x$
	will be found, and will be updated according to the following formula:
	
		$$i(x)^* = i(x) + \eta f(d) (x - i(x))),$$
	
	where $\eta$ is the learning rate, $d$ is the distance, and $f(d)$ is:
	
		$$f(d) = \exp\left(-\frac{d^2}{2\sigma^2}\right).$$
}

\frame {
	\frametitle{Brief Revision of SOM (cont.)}
	
	The neighborhood is determined according to $\sigma$, which will be updated after each iteration:
	
		$$\sigma(n) = \sigma_0 \exp\left(-\frac{n}{\tau}\right),$$
	
	$\tau$ is a factor chosen according to our needs.
	
}

\section{Data Structure in SOM-SD}
\frame {
	\frametitle{Data Structure in SOM-SD}
	
	The data structure considered in SOM-SD are \blue{labeled DAGs}, where labels are tuples of variables and are attached to
	vertices and possibly to edges. The attached values represents the relationship between the vertices.\\
	
	For a data structure $\mathcal{D}$, let $c$ be the maximum out degree of the corresponding DAG, then the labels $r$ will have
	value tuple in the following form:
	
		$$\mathbb{R}^m \times \underbrace{\mathbb{R}^n \times \cdots \times \mathbb{R}^n}_{c \textsf{times}}.$$
	
	The tuple will contain the infomation of the children of current node $v$, that's why we have $c$ terms of $\mathbb{R}^n$ -- they are
	the coordination of $v$'s children, and the blanks will be filled by a pre-defined vector such as
	$\underbrace{(-1, \ldots, -1)}_{n \textsf{times}}.$
}

\section{Algorithm of SOM-SD}

\frame {
	\frametitle{Algorithm of SOM-SD}
	
	TODO
	
}

\frame {

	\frametitle{Algorithm of SOM-SD (cont.)}
	
	\begin{center}
		%\includegraphics[width=150pt]{som-sd-example.jpg}
	\end{center}
}

\section {My Experiment Result}
\frame {
	
}

\section {Challenges to SOM-SD}
\frame {
	\frametitle{Challenges to SOM-SD}
	
	A potential challenge to SOM-SD is that it is probably unstable to changes of input data structure.
	Consider the possibility that input data structure $S$ has multiple representations $S_1, S_2, \ldots$,
	but they are slightly different from each other. Suppose the SOM-SD is trained to recognize $S_1$ correctly,
	then we change $S$ to the representation of $S_2$, then the SOM-SD will probably fail.
}

\frame {
	\begin{center}
		%\includegraphics[width=200pt]{niceboat.png}\\
		\small{A challenge to SOM-SD}
	\end{center}
}

\end{document}
