\documentclass[adobefonts,UTF8,a4paper,fancyhdr,fntef]{ctexart}

\usepackage[xetex]{hyperref}
\usepackage{listings}
\usepackage{enumerate}
\usepackage[margin=3cm]{geometry}
\usepackage{color}
\definecolor{kwdclr}{rgb}{0.0, 0.0, 0.7}
\newcommand{\kwdsty}{\color{kwdclr}\bfseries}
\usepackage{fontspec,xunicode}
\setCJKmainfont[BoldFont={SimHei}, ItalicFont={KaiTi_GB2312}]{SimSun}
\setCJKsansfont{SimHei}
\setCJKmonofont{KaiTi_GB2312}
\setmainfont{Baskerville}
\setsansfont{Arial}
\setmonofont{Courier New}

\makeatletter
\def\maketitle{  
\thispagestyle{empty}  
\begin{center}
	\normalfont  
	\vspace*{3cm}
     	{\LARGE\bf \@title\par}
    	 \vskip 2mm  
     	{\large \@author\par}
     	\vskip 1mm  
     	{\large \@date\par}
\end{center}
}  
\renewcommand{\section}{\@startsection{section}{1}{-1em}{-\baselineskip}{0.5\baselineskip}{\Large\bf\leftline}}
\renewcommand{\thesection}{}

\renewcommand{\subsection}{\@startsection{subsection}{2}{-1em}{-0.5\baselineskip}{0.3\baselineskip}{\large\bf\leftline}}
\renewcommand{\thesubsection}{}

\renewcommand{\subsubsection}{\@startsection{subsubsection}{3}{1em}{-0.2\baselineskip}{0.1\baselineskip}{\bf\leftline}}
\renewcommand{\thesubsubsection}{}
\makeatother

\begin{document}

\title{算法第七次作业}
\author{2010210798 叶丰}
\maketitle

\section{15.1-1}
代码如下:


\section{15.1-4}
可以看到,FASTEST-WAY函数中的$f_i[j]$表格在计算的时候,$f_i[j]$的值只依赖于$f_{i-1}[j]$,且计算过程是一直向前的。这意味着$f_i[j]$的空间可以从$2n$变为$4$,即每次只存储正在计算的$f_i[j]$和它的上一步$f_{i-1}[j]$。这样$f_i[j]$和$l_i[j]$的空间就从$4n-2$变成了$2n+2$。

\section{15.2-1}
计算结果如图:
\begin{figure}[H]
\centering

\end{figure}
\begin{figure}[H]
\centering

\end{figure}

最少的乘法次数为$2010$,方案是$(1\ 2)((3\ 4)(5\ 6))$。

\section{15.3-3}
具有最优子结构。对给定的矩阵序列,按照其最优的加括号方案,用最外层括号将序列分成两段,则这两段内部的加括号方案一定是最优的。否则,可以用更优的方案来替换原有方案,到达更优的解。

\section{15.3-4}
在计算$f_1[j]$时,会用到$f_1[j-1]$和$f_2[j-1]$,同时在计算$f_2[j]$时,又会再次用到$f_1[j-1]$和$f_2[j-1]$。因此出现了一个子问题需要计算两次的情况,即重叠子问题。

\section{15.5-2}
计算结果如图:
\begin{figure}[H]
\centering

\end{figure}
\begin{figure}[H]
\centering

\end{figure}
\begin{figure}[H]
\centering

\end{figure}

最优二叉查找树的代价为$3.12$,结构为:
\begin{figure}[H]
\centering

\end{figure}

\section{15.4}
算法伪代码如下:


上面算法中,使用了每个结点的$rating$值,它需要满足约喜欢聚会的人$rating$越高,这可以从排名的倒数乘以某个常数来得到。

上面的算法的基本思想是:对树进行深度优先搜索,并且当某个节点的所有子结点都被搜索完成后(即所有子结点变黑和本结点变黑之间),根据其子结点的最优值得到本结点的最优值。最优值有两种情况:1) 本结点不参加聚会的最优值$not\mbox{-}chosen[x]$是以各个子结点为根的子树的最优值$optimal$之和,2) 本结点参加聚会的最优值$chosen[x]$是以各个子结点为根的子树的$not\mbox{-}chosen$值之和加上本结点的$rating$值。最后得到以某个结点为根的子树的最优值是$not\mbox{-}chosen$和$chosen$值的较大值,存在结点的$optimal$中。

经过上述计算后,就可以输出最优解,即参加聚会的名单了。伪代码如下:


直接对根结点调用PRINT-LIST,并且将$chooseable$置为TRUE即可以打印出聚会名单。

因为每个结点至多被调用OPTIMAL-PARTY一次,并且每次调用中,repeat-until中的循环只会对结点的所有子结点进行一次访问,故算法的执行时间是$O(n)$,$n$是结点个数。

\section{15.7}
首先对$a_1,a_2,\cdots,a_n$按照$d[i]$的递增顺序排序。方便起见,不妨设排序后的顺序就是原顺序,即有$d[i-1]\leq d[i]$。算法的伪代码如下:


输出最优调度的伪代码:


下面考虑算法的复杂度。注意到每个作业的运行时间不超过$n$,则可以认为作业的$d[i]$均不超过$n^2$。否则,如果存在某个$d[i]>n^2$,可以将$d[i]$变为$n^2$而不影响计算结果。故算法复杂度为$O(n^3)$。

\end{document}
