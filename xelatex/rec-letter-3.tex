%\documentclass{oxmathletter} % if using pdflatex
\documentclass[xelatex]{styles/rec-letter-3} % if using xelatex 

\signature{Frank R. Action} % your name as you want it to appear as signature

% This document class should only be used for official departmental business
% As such the option defaults are the official departmental general contact
% details. Defaults should only be overridden with your personal official 
% departmental contact details

%%%%%%%%%%%%%%%%%%%%%%%%%%%%%%%%%%%%%%%%%%%%%%%%%%%%%%%%%%%%%%%%%%%%%%%%%%%%
%optional commands to override defaults
%do not include these commands if you wish to stick with the defaults
\position{Head of Sums} % your role/position, default none shown

\phone{251364}   % your direct phone number (area code is added automatically)
                  % default none shown

\fax{270515}     % fax number (area code is added automatically)
                  % defaults to St Giles reception fax

\econtact{frank.action}  % your official (long format) maths e-mail address 
                         % without @maths.ox.ac.uk 
                         % this will produce your official email address
                         % and the URL of your official web profile page

\refcode{MIIT/2010/17A} % optional reference code to specify on letter
                         % default none shown

%end of optional commands
%%%%%%%%%%%%%%%%%%%%%%%%%%%%%%%%%%%%%%%%%%%%%%%%%%%%%%%%%%%%%%%%%%%%%%%%%%%%

\begin{document}

\begin{letter}{
Prof. Earnest P. Silon \\  
Dept of Mathematics\\
University of Netherfield\\
NF1 2QT}

\subject{Departmental \LaTeX{} Letter Class similar to Word Version}  
% optional subject line

\opening{Dear Earnest,}  % initial greeting/opening
To whom it may concern:

I deem great pleasure to recommend one of my best students, Mr. Yang Zhang, as a promising candidate for your graduate program.

I have known Yang since 2005, when I was appointed as teacher in charge of his class. During the undergraduate years, Yang proved himself to be a bright, diligent student. He completed his Bachelor degree with an excellent GPA of 88/100, ranking top 20\% among more than 160 students. He had chosen the most difficult courses, including Operating Systems, Theoretical Computer Science, Database Implementation, etc. His experience in those courses exhibits outstanding coding skills, and a deep interest in fundamentals of computer technology. Based on those observations, I enrolled him into our institution.

Yang did excellent in graduate school, too. He earned a 94/100 score in my "Algorithm and Its Complexity" course, ranking 3rd out of about 180 students. Next year, upon his request, I happily offered him as my TA on this course. And it was then that I knew he had already worked as TA for "Java Programming Language" back in 2008, when he was still a undergraduate student. I was very impressed, since TA in our department was only hired among Master and PhD students. A undergraduate student working as TA was unheard of, which showed that he must have done extraordinarily well in that course, and was highly appreciated by the teacher. He proved himself a great TA in my course, carefully layout homework solutions with LaTeX, organized revision seminars, and helped to prepare final exam questions. I highly appreciated his attitudes and sense of responsibility, and invited him to work as TA for my another course, "Experiments on High Performance Computing". He gave a talk on building distributed key-value storage systems with Apache Cassandra to more than 40 students.

Yang have been working diligently in our lab. There are 3 groups in our lab, and he is leader of one group. He worked in the lab almost every day, and often late into the night. He is experienced in giving talks. First year graduate students in our lab is required to give talks on book chapters from Andrew S. Tanenbaum's book, "Distributed Systems: Principles and Paradigms". Yang could grasp the gist of the book chapters quickly, and express that to fellow students. And, after the lecture series ended, Yang continued to organize such seminars with lab members, sharing knowledge with each other.



Sincerely yours,


Dr. Yongwei Wu

Professor
Deputy director
Institute of High Performance Computing of Tsinghua University

\closing{Yours sincerely} % closing phrase
\end{letter}
\end{document}
